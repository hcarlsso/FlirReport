\section{Conclusion}

A mathematical model for the temperature evolution of a microbolometer
and the corresponding output voltage was derived. The model is built
using the heat equation, the Stefan Boltzmann's law, and the Joule
heating law. The change in the temperature of the bolometer
is measured through the resistance change. The heat equation model was
modified based on heuristic arguments to include an additive noise
term in the bias voltage, and the resulting differential equation was
discretized and solved using the explicit Euler scheme.

This, however, led to counter intuitive results regarding the NETD and
the power spectrum of the output voltage, raising the question
whenever the assumed stochastic model was accurate
enough. Unfortunately, there was no experimental data available to
verify the numerical simulations to.

Nevertheless, hindsight provided us with a more well defined
stochastic model to model the phenomena. Unfortunately, time was not
enough to try out this idea.


%%% Local Variables:
%%% mode: latex
%%% TeX-master: "main"
%%% TeX-PDF-mode: 1
%%% TeX-PDF-via-dvips-ps2pdf: 1
%%% End:
