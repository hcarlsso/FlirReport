\section{Outlook and future extensions}

\subsection{Noise process inclusion}
In section 3 we went through the heuristic underlying the inclusion of
noise in the model. This heuristic is however not entirely
satisfactory. Although one can certainly argue for the negligibility
of the squared noise term, it was dropped rather ad hoc. Neither did
we account for the fact that the noise as inserted in the heuristics
is subject to multiplication by a $dt$-term in the actual simulation,
making it hard to establish a scaling scheme for the noise term with a
well defined stochastic limit as step-size goes to 0. What we need in
order to fully solve this matter is stochastic calculus.


Consider thermal noise and assume that the bias voltage $V(t)$ can be
described as a diffusion process. Thermal noise has a white noise
behavior, so ideally, stochasticity around the mean of the bias
voltage $V_{m}(t)$ should have white noise form. Unfortunately, white
noise is not possible to write as a diffusion process. We therefore
suggest that $V(t)$ is modeled as follows:


\begin{align*}
d V = k(V_{m}(t)-V(t)) + \sigma dW_{t}
\end{align*}

This is a Ornstein-Uhlenbeck process, which imposes a mean reverting
mechanism. If the process is above $V_m(t)$, the drift term is
negative and the process will tend to move towards $V_m(t)$. If the
process is below $V_m(t)$, the reverse holds true. This makes
correlation and variance increase over time limited when comparing to the
standard Brownian e.g. By adjusting the constants $\sigma$ and $k$, we
can generate an erratic process dynamics with mean-adjusted noise much
resembling white noise.


Let us consider what happens with the temperature dynamics in
equation~\eqref{eq:heat_balance_equation} when we assume the function
$V(t)$ has the stochastic dynamics as described above. We are going to
use the Lemma of Ito, which states that if a diffusion process $X$ has
dynamics $dX = \mu(t)dt + \sigma(t)dW_{t}$, then $f(X)$ (under certain
regularity conditions) has dynamics


\begin{align*}
d f = \left(\frac{\partial f}{ \partial t} + \mu(t)\frac{\partial f}{ \partial x} + \frac{\sigma(t)^{2}}{2}\frac{\partial^{2} f}{ \partial^{2} x}\right)dt + \sigma(t)\frac{\partial f}{ \partial x}dW_{t}
\end{align*}

In our case $f(x) = x^{2}$, so $\frac{\partial f}{ \partial t} = 0$, $\frac{\partial f}{ \partial x} = 2x$, $\frac{\partial^{2} f}{ \partial^{2} x} = 2$. Moreover, $\mu(t) = K(V_{m}(t)-V(t))$, and $\sigma(t)=\sigma$.
Plugging in these values in the Ito formula gives

\begin{align*}
d f = ( k(V_{m}(t)-V(t))2V(t) + \sigma^{2})dt + \sigma(t)2V(t)dW_{t}
\end{align*}

or

\begin{align*}
d X(t)= ( k(V_{m}(t)-\sqrt{X(t)})2\sqrt{X(t)} + \sigma^{2})dt + \sigma(t)2\sqrt{X(t)}dW_{t}
\end{align*}

Note that the temperature dynamics constitute a well defined
stochastic process with the deterministic $V$  replaced by the
stochastic function $X$. The replacement operation turns the
deterministic ODE into a diffusion process with another diffusion
process ($X$) as drift coefficient. As $X$ is adapted to the same
underlying filter as $T$, the process is well defined via the Ito
integral.


%%% Local Variables:
%%% mode: latex
%%% TeX-master: "main"
%%% TeX-PDF-mode: 1
%%% TeX-PDF-via-dvips-ps2pdf: 1
%%% End:
