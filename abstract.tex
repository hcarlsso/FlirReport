\begin{abstract}
A microbolometer is a device for measuring the power of incident electromagnetic radiation via the heating of a material with a temperature-dependent electrical resistance in infrared (IR) cameras. In the camera, an array of bolometers gives an IR image. In T\"aby, FLIR develops and produce cameras for temperature measurement. Each pixel of the IR image is represented by a readout voltage, which is a noisy signal. This voltage is the signal that is measurable at the moment. However, in the literature, there are mathematical models describing the underlying behavior of the bolometer, before the readout voltage. We would like to investigate if the underlying bolometer parameters can be identified using readout voltage data. It is also interesting to get an understanding of how the properties of the parameter estimates are affected by noise. A better understanding of the components (detectors) could potentially strengthen the capabilities to reduce noise. 
\end{abstract}
