\begin{abstract}
A microbolometer is a device for measuring the power of incident
electromagnetic radiation via the heating of a material with a
temperature-dependent electrical resistance in infrared (IR)
cameras. The change in the resistance is measured with an applied bias
voltage, which yields a current that is fed to an integrator. The
integrator then yields a readout voltage that represents the output
signal of the system. The bias voltage also heats the resistance, and
thus in the uncooled microbolometer system, the bias voltage is only applied
periodically to allow the resistance to cool down. In the camera, an
array of bolometers gives an IR image. Based on the heat equation and
the Stefan Boltzmann law of black-body radiation, models can describe
the underlying behavior of the
bolometer, before the readout voltage. We would like to investigate if
the underlying bolometer parameters can be identified using readout
voltage data. Further, we investigate how noise affects the system and
how the models can be modified to account for the noise. A
better understanding of the components and the noise phenomena could potentially
yield better detectors. This report is done in
collaboration with FLIR, based in T\"aby. FLIR develops and produces
cameras for temperature
measurement.

\end{abstract}

%%% Local Variables:
%%% mode: latex
%%% TeX-master: "main"
%%% TeX-PDF-mode: 1
%%% TeX-PDF-via-dvips-ps2pdf: 1
%%% End:
